% Generated by Sphinx.
\def\sphinxdocclass{report}
\documentclass[a4paper,10pt,french]{sphinxmanual}
\usepackage[utf8]{inputenc}
\DeclareUnicodeCharacter{00A0}{\nobreakspace}
\usepackage{cmap}
\usepackage[T1]{fontenc}
\usepackage[french]{babel}
\usepackage{times}
\usepackage[Sonny]{fncychap}
\usepackage{longtable}
\usepackage{sphinx}
\usepackage{multirow}

\addto\captionsfrench{\renewcommand{\figurename}{Fig. }}
\addto\captionsfrench{\renewcommand{\tablename}{Tableau }}
\floatname{literal-block}{Code source }



\title{predictif Documentation}
\date{03 April 2015}
\release{0.1}
\author{H. Verlin - S. Toko}
\newcommand{\sphinxlogo}{}
\renewcommand{\releasename}{Version}
\makeindex

\makeatletter
\def\PYG@reset{\let\PYG@it=\relax \let\PYG@bf=\relax%
    \let\PYG@ul=\relax \let\PYG@tc=\relax%
    \let\PYG@bc=\relax \let\PYG@ff=\relax}
\def\PYG@tok#1{\csname PYG@tok@#1\endcsname}
\def\PYG@toks#1+{\ifx\relax#1\empty\else%
    \PYG@tok{#1}\expandafter\PYG@toks\fi}
\def\PYG@do#1{\PYG@bc{\PYG@tc{\PYG@ul{%
    \PYG@it{\PYG@bf{\PYG@ff{#1}}}}}}}
\def\PYG#1#2{\PYG@reset\PYG@toks#1+\relax+\PYG@do{#2}}

\expandafter\def\csname PYG@tok@s1\endcsname{\def\PYG@tc##1{\textcolor[rgb]{0.25,0.44,0.63}{##1}}}
\expandafter\def\csname PYG@tok@nd\endcsname{\let\PYG@bf=\textbf\def\PYG@tc##1{\textcolor[rgb]{0.33,0.33,0.33}{##1}}}
\expandafter\def\csname PYG@tok@vg\endcsname{\def\PYG@tc##1{\textcolor[rgb]{0.73,0.38,0.84}{##1}}}
\expandafter\def\csname PYG@tok@cp\endcsname{\def\PYG@tc##1{\textcolor[rgb]{0.00,0.44,0.13}{##1}}}
\expandafter\def\csname PYG@tok@kt\endcsname{\def\PYG@tc##1{\textcolor[rgb]{0.56,0.13,0.00}{##1}}}
\expandafter\def\csname PYG@tok@c1\endcsname{\let\PYG@it=\textit\def\PYG@tc##1{\textcolor[rgb]{0.25,0.50,0.56}{##1}}}
\expandafter\def\csname PYG@tok@gd\endcsname{\def\PYG@tc##1{\textcolor[rgb]{0.63,0.00,0.00}{##1}}}
\expandafter\def\csname PYG@tok@mb\endcsname{\def\PYG@tc##1{\textcolor[rgb]{0.13,0.50,0.31}{##1}}}
\expandafter\def\csname PYG@tok@kd\endcsname{\let\PYG@bf=\textbf\def\PYG@tc##1{\textcolor[rgb]{0.00,0.44,0.13}{##1}}}
\expandafter\def\csname PYG@tok@m\endcsname{\def\PYG@tc##1{\textcolor[rgb]{0.13,0.50,0.31}{##1}}}
\expandafter\def\csname PYG@tok@sx\endcsname{\def\PYG@tc##1{\textcolor[rgb]{0.78,0.36,0.04}{##1}}}
\expandafter\def\csname PYG@tok@no\endcsname{\def\PYG@tc##1{\textcolor[rgb]{0.38,0.68,0.84}{##1}}}
\expandafter\def\csname PYG@tok@mi\endcsname{\def\PYG@tc##1{\textcolor[rgb]{0.13,0.50,0.31}{##1}}}
\expandafter\def\csname PYG@tok@se\endcsname{\let\PYG@bf=\textbf\def\PYG@tc##1{\textcolor[rgb]{0.25,0.44,0.63}{##1}}}
\expandafter\def\csname PYG@tok@sb\endcsname{\def\PYG@tc##1{\textcolor[rgb]{0.25,0.44,0.63}{##1}}}
\expandafter\def\csname PYG@tok@c\endcsname{\let\PYG@it=\textit\def\PYG@tc##1{\textcolor[rgb]{0.25,0.50,0.56}{##1}}}
\expandafter\def\csname PYG@tok@w\endcsname{\def\PYG@tc##1{\textcolor[rgb]{0.73,0.73,0.73}{##1}}}
\expandafter\def\csname PYG@tok@go\endcsname{\def\PYG@tc##1{\textcolor[rgb]{0.20,0.20,0.20}{##1}}}
\expandafter\def\csname PYG@tok@k\endcsname{\let\PYG@bf=\textbf\def\PYG@tc##1{\textcolor[rgb]{0.00,0.44,0.13}{##1}}}
\expandafter\def\csname PYG@tok@s\endcsname{\def\PYG@tc##1{\textcolor[rgb]{0.25,0.44,0.63}{##1}}}
\expandafter\def\csname PYG@tok@kn\endcsname{\let\PYG@bf=\textbf\def\PYG@tc##1{\textcolor[rgb]{0.00,0.44,0.13}{##1}}}
\expandafter\def\csname PYG@tok@sc\endcsname{\def\PYG@tc##1{\textcolor[rgb]{0.25,0.44,0.63}{##1}}}
\expandafter\def\csname PYG@tok@gh\endcsname{\let\PYG@bf=\textbf\def\PYG@tc##1{\textcolor[rgb]{0.00,0.00,0.50}{##1}}}
\expandafter\def\csname PYG@tok@nn\endcsname{\let\PYG@bf=\textbf\def\PYG@tc##1{\textcolor[rgb]{0.05,0.52,0.71}{##1}}}
\expandafter\def\csname PYG@tok@o\endcsname{\def\PYG@tc##1{\textcolor[rgb]{0.40,0.40,0.40}{##1}}}
\expandafter\def\csname PYG@tok@ne\endcsname{\def\PYG@tc##1{\textcolor[rgb]{0.00,0.44,0.13}{##1}}}
\expandafter\def\csname PYG@tok@mf\endcsname{\def\PYG@tc##1{\textcolor[rgb]{0.13,0.50,0.31}{##1}}}
\expandafter\def\csname PYG@tok@nl\endcsname{\let\PYG@bf=\textbf\def\PYG@tc##1{\textcolor[rgb]{0.00,0.13,0.44}{##1}}}
\expandafter\def\csname PYG@tok@mo\endcsname{\def\PYG@tc##1{\textcolor[rgb]{0.13,0.50,0.31}{##1}}}
\expandafter\def\csname PYG@tok@sd\endcsname{\let\PYG@it=\textit\def\PYG@tc##1{\textcolor[rgb]{0.25,0.44,0.63}{##1}}}
\expandafter\def\csname PYG@tok@nc\endcsname{\let\PYG@bf=\textbf\def\PYG@tc##1{\textcolor[rgb]{0.05,0.52,0.71}{##1}}}
\expandafter\def\csname PYG@tok@ni\endcsname{\let\PYG@bf=\textbf\def\PYG@tc##1{\textcolor[rgb]{0.84,0.33,0.22}{##1}}}
\expandafter\def\csname PYG@tok@gs\endcsname{\let\PYG@bf=\textbf}
\expandafter\def\csname PYG@tok@sh\endcsname{\def\PYG@tc##1{\textcolor[rgb]{0.25,0.44,0.63}{##1}}}
\expandafter\def\csname PYG@tok@gu\endcsname{\let\PYG@bf=\textbf\def\PYG@tc##1{\textcolor[rgb]{0.50,0.00,0.50}{##1}}}
\expandafter\def\csname PYG@tok@mh\endcsname{\def\PYG@tc##1{\textcolor[rgb]{0.13,0.50,0.31}{##1}}}
\expandafter\def\csname PYG@tok@ss\endcsname{\def\PYG@tc##1{\textcolor[rgb]{0.32,0.47,0.09}{##1}}}
\expandafter\def\csname PYG@tok@kp\endcsname{\def\PYG@tc##1{\textcolor[rgb]{0.00,0.44,0.13}{##1}}}
\expandafter\def\csname PYG@tok@kc\endcsname{\let\PYG@bf=\textbf\def\PYG@tc##1{\textcolor[rgb]{0.00,0.44,0.13}{##1}}}
\expandafter\def\csname PYG@tok@ge\endcsname{\let\PYG@it=\textit}
\expandafter\def\csname PYG@tok@kr\endcsname{\let\PYG@bf=\textbf\def\PYG@tc##1{\textcolor[rgb]{0.00,0.44,0.13}{##1}}}
\expandafter\def\csname PYG@tok@il\endcsname{\def\PYG@tc##1{\textcolor[rgb]{0.13,0.50,0.31}{##1}}}
\expandafter\def\csname PYG@tok@vc\endcsname{\def\PYG@tc##1{\textcolor[rgb]{0.73,0.38,0.84}{##1}}}
\expandafter\def\csname PYG@tok@na\endcsname{\def\PYG@tc##1{\textcolor[rgb]{0.25,0.44,0.63}{##1}}}
\expandafter\def\csname PYG@tok@sr\endcsname{\def\PYG@tc##1{\textcolor[rgb]{0.14,0.33,0.53}{##1}}}
\expandafter\def\csname PYG@tok@ow\endcsname{\let\PYG@bf=\textbf\def\PYG@tc##1{\textcolor[rgb]{0.00,0.44,0.13}{##1}}}
\expandafter\def\csname PYG@tok@gr\endcsname{\def\PYG@tc##1{\textcolor[rgb]{1.00,0.00,0.00}{##1}}}
\expandafter\def\csname PYG@tok@gp\endcsname{\let\PYG@bf=\textbf\def\PYG@tc##1{\textcolor[rgb]{0.78,0.36,0.04}{##1}}}
\expandafter\def\csname PYG@tok@nf\endcsname{\def\PYG@tc##1{\textcolor[rgb]{0.02,0.16,0.49}{##1}}}
\expandafter\def\csname PYG@tok@cm\endcsname{\let\PYG@it=\textit\def\PYG@tc##1{\textcolor[rgb]{0.25,0.50,0.56}{##1}}}
\expandafter\def\csname PYG@tok@err\endcsname{\def\PYG@bc##1{\setlength{\fboxsep}{0pt}\fcolorbox[rgb]{1.00,0.00,0.00}{1,1,1}{\strut ##1}}}
\expandafter\def\csname PYG@tok@gt\endcsname{\def\PYG@tc##1{\textcolor[rgb]{0.00,0.27,0.87}{##1}}}
\expandafter\def\csname PYG@tok@vi\endcsname{\def\PYG@tc##1{\textcolor[rgb]{0.73,0.38,0.84}{##1}}}
\expandafter\def\csname PYG@tok@cs\endcsname{\def\PYG@tc##1{\textcolor[rgb]{0.25,0.50,0.56}{##1}}\def\PYG@bc##1{\setlength{\fboxsep}{0pt}\colorbox[rgb]{1.00,0.94,0.94}{\strut ##1}}}
\expandafter\def\csname PYG@tok@nt\endcsname{\let\PYG@bf=\textbf\def\PYG@tc##1{\textcolor[rgb]{0.02,0.16,0.45}{##1}}}
\expandafter\def\csname PYG@tok@gi\endcsname{\def\PYG@tc##1{\textcolor[rgb]{0.00,0.63,0.00}{##1}}}
\expandafter\def\csname PYG@tok@s2\endcsname{\def\PYG@tc##1{\textcolor[rgb]{0.25,0.44,0.63}{##1}}}
\expandafter\def\csname PYG@tok@si\endcsname{\let\PYG@it=\textit\def\PYG@tc##1{\textcolor[rgb]{0.44,0.63,0.82}{##1}}}
\expandafter\def\csname PYG@tok@nv\endcsname{\def\PYG@tc##1{\textcolor[rgb]{0.73,0.38,0.84}{##1}}}
\expandafter\def\csname PYG@tok@bp\endcsname{\def\PYG@tc##1{\textcolor[rgb]{0.00,0.44,0.13}{##1}}}
\expandafter\def\csname PYG@tok@nb\endcsname{\def\PYG@tc##1{\textcolor[rgb]{0.00,0.44,0.13}{##1}}}

\def\PYGZbs{\char`\\}
\def\PYGZus{\char`\_}
\def\PYGZob{\char`\{}
\def\PYGZcb{\char`\}}
\def\PYGZca{\char`\^}
\def\PYGZam{\char`\&}
\def\PYGZlt{\char`\<}
\def\PYGZgt{\char`\>}
\def\PYGZsh{\char`\#}
\def\PYGZpc{\char`\%}
\def\PYGZdl{\char`\$}
\def\PYGZhy{\char`\-}
\def\PYGZsq{\char`\'}
\def\PYGZdq{\char`\"}
\def\PYGZti{\char`\~}
% for compatibility with earlier versions
\def\PYGZat{@}
\def\PYGZlb{[}
\def\PYGZrb{]}
\makeatother

\renewcommand\PYGZsq{\textquotesingle}

\begin{document}

\maketitle
\tableofcontents
\phantomsection\label{index::doc}


\emph{Prédict’IF} est un cabinet de voyance en ligne, qui envoie périodiquement des email à ses clients. Ces emails contiennent des horoscopes personnalisés créés par les employés de l'entreprise. Ceux-ci se basent sur le signe du client, et sur le choix des voyants qu'il a séléctionné lors de son inscription.

Les services présentés ici permettent a création des différentes briques de l'application Prédict’IF, c'est-à-dire à la réalisation des différentes IHM.

L'ensemble des services fournis permet la réalisation de deux IHM :
\begin{itemize}
\item {} 
Une IHM web pour les clients, qui leur permet de s'inscrire pour recevoir des horoscopes

\item {} 
Une IHM web pour les employés, pour la réalisation des horoscopes

\end{itemize}

Vous trouverez dans cette documentation :
\begin{enumerate}
\item {} 
Le modèle du domaine avec un diagramme de classe UML

\item {} 
La description des services

\item {} 
La maquette des 3 IHM (président, employés, clients)

\end{enumerate}

\begin{notice}{tip}{Astuce:}
Pour une lecture plus agréable de la doc aller sur \href{http://predictif-doc.readthedocs.org/}{http://predictif-doc.readthedocs.org/}
\end{notice}


\chapter{Modèle du domaine}
\label{modele:predictif-documentation}\label{modele:modele-du-domaine}\label{modele::doc}

\section{Liste des entity}
\label{modele:liste-des-entity}
L'analyse des besoins concernant cette application nous a permis de définir ces différentes classes :

\begin{tabulary}{\linewidth}{|L|L|}
\hline
\textsf{\relax 
Nom
} & \textsf{\relax 
Fonction
}\\
\hline
Client
 & 
Client de predict'if
\\
\hline
Employe
 & 
Employe de Predict'if
\\
\hline
Medium
 & 
Medium fictif
\\
\hline
Horoscope
 & 
Horoscope contenant un ensemble de 3 prédictions de type différent
\\
\hline
Prédiction
 & 
Classe abstraite qui sert de base pour les 3 types de prédictions
\\
\hline
PrédictionAmour
 & 
Prédiction de catégorie Amour,
indique un signe partenaire
\\
\hline
PrédictionTravail
 & 
Prédiction de catégorie Travail
\\
\hline
PrédictionSante
 & 
Prediction de catégorie Santé, contient un conseil pour le client
\\
\hline
Signe
 & 
Réprésente un signe astrologique simplifié
(correspond à un mois de l'année)
\\
\hline\end{tabulary}


L'ensemble des classes dispose d'une méthode \code{toString()} pour permettre son affichage.


\section{Diagramme de Classes}
\label{modele:diagramme-de-classes}

\chapter{Description des services}
\label{services:description-des-services}\label{services::doc}

\section{Méthodes de classe \texttt{Services}}
\label{services:methodes-de-classe-services}
\begin{tabulary}{\linewidth}{|p{6cm}|p{4cm}|p{5cm}|}
\hline
\textsf{\relax 
Méthode
} & \textsf{\relax 
Type de Retour
} & \textsf{\relax 
Brève Description
}\\
\hline
creerClient(Client client)
 & 
int
 & 
Permet d'insérer un client dans la base de données
\\
\hline
creerEmploye(Employe employe)
 & 
void
 & 
Permet d'insérer un employé dans la base de données
\\
\hline
creerHoroscope(Horoscope horoscope, Client client)
 & 
void
 & 
Permet d'insérer un horoscope dans la base de données,
et de l'attacher à un client.
Elle simule l'envoi du mail au client
\\
\hline
creerMedium(Medium medium)
 & 
void
 & 
Permet d'insérer un signe dans la base de données
\\
\hline
creerPredictionAmour(PredictionAmour predictionAmour)
 & 
void
 & 
Permet d'insérer une prédiction de
type amour dans la base de données
\\
\hline
creerPredictionSante(PredictionSante predictionSante)
 & 
void
 & 
Permet d'insérer une prédiction
de type santé dans la base de données
\\
\hline
creerPredictionTravail(PredictionTravail predictionTravail)
 & 
void
 & 
Permet d'insérer une prédiction de type
travail dans la base de données
\\
\hline
creerSigne(Signe signe)
 & 
void
 & 
Permet d'insérer un signe dans la base de données
\\
\hline
getClientById(long id)
 & 
Client
 & 
Récupère un client grâce à son id
\\
\hline
getEmployeById(long id)
 & 
Employe
 & 
Récupère un employe grâce à son id
\\
\hline
getSigneById(long id)
 & 
Signe
 & 
Récupère un signe grâce à son id
\\
\hline
getPredictionById(long id)
 & 
Prediction
 & 
Récupère un prediction grâce à son id
\\
\hline
getHoroscopeById(long id)
 & 
Horoscope
 & 
Récupère un horoscope grâce à son id
\\
\hline
getMediumById(long id)
 & 
Medium
 & 
Récupère un medium grâce à son id
\\
\hline
listerClients()
 & 
List\textless{}Client\textgreater{}
 & 
Liste de toutes les clients de la base
\\
\hline
listerHistoriqueClient(Client client)
 & 
List\textless{}Horoscope\textgreater{}
 & 
Renvoie l'historique de l'horoscope d'un client
\\
\hline
listerMediums()
 & 
List\textless{}Medium\textgreater{}
 & 
Liste de tous les mediums de la base
\\
\hline
listerMediumsClient(Client client)
 & 
List\textless{}Medium\textgreater{}
 & 
Liste tous les mediums d'un client
\\
\hline
listerPredictionAmours()
 & 
List\textless{}PredictionAmour\textgreater{}
 & 
Liste de toutes les predictions amour de la base
\\
\hline
listerPredictionSantes()
 & 
List\textless{}PredictionSante\textgreater{}
 & 
Liste de toutes les predictions sante de la base
\\
\hline
listerPredictionTravails()
 & 
List\textless{}PredictionTravail\textgreater{}
 & 
Liste de toutes les predictions travail de la base
\\
\hline
listerSignes()
 & 
List\textless{}Signe\textgreater{}
 & 
Liste de toutes les signes de la base
\\
\hline
updateEmp(Employe emp)
 & 
void
 & 
Met à jour un employé dans la base
\\
\hline
connexionEmploye(String identifiant, String mdp)
 & 
Employe
 & 
Connexion d'un employé
\\
\hline
listerMediumsClient(Client client)
 & 
List\textless{}Medium\textgreater{}
 & 
Liste tous les mediums d'un client
\\
\hline\end{tabulary}



\section{Exemples d'utilisation}
\label{services:exemples-d-utilisation}

\subsection{Créer un client}
\label{services:creer-un-client}
\begin{notice}{tip}{Astuce:}
La class \href{http://docs.oracle.com/javase/7/docs/api/java/util/Calendar.html}{Calendar} est utilisée pour la gestion des dates.
\end{notice}

\begin{Verbatim}[commandchars=\\\{\}]
\PYG{c+c1}{// on imagine que le client est né aujourd\PYGZsq{}hui}
\PYG{n}{Calendar} \PYG{n}{DateNaissance} \PYG{o}{=} \PYG{n}{Calendar}\PYG{o}{.}\PYG{n+na}{getInstance}\PYG{o}{(}\PYG{o}{)}\PYG{o}{;}
\PYG{n}{Client} \PYG{n}{c} \PYG{o}{=} \PYG{k}{new} \PYG{n}{Client}\PYG{o}{(}\PYG{l+s}{\PYGZdq{}Nom\PYGZdq{}}\PYG{o}{,} \PYG{l+s}{\PYGZdq{}Prénom\PYGZdq{}}\PYG{o}{,}\PYG{l+s}{\PYGZdq{}Mr\PYGZdq{}}\PYG{o}{,} \PYG{n}{DateNaissance}\PYG{o}{,}
        \PYG{l+s}{\PYGZdq{}rue de nullepart, 69 100 If\PYGZus{}laville\PYGZdq{}}\PYG{o}{,} \PYG{l+s}{\PYGZdq{}client@if.fr\PYGZdq{}}\PYG{o}{,}\PYG{l+s}{\PYGZdq{}06 26 30 29 \PYGZdq{}}\PYG{o}{)}\PYG{o}{;}
\PYG{n}{services}\PYG{o}{.}\PYG{n+na}{creerClient}\PYG{o}{(}\PYG{n}{c}\PYG{o}{)}\PYG{o}{;}
\end{Verbatim}


\subsection{Connexion d'un employé}
\label{services:connexion-d-un-employe}
\begin{Verbatim}[commandchars=\\\{\}]
\PYG{n}{Employe} \PYG{n}{employe} \PYG{o}{=} \PYG{n}{services}\PYG{o}{.}\PYG{n+na}{connexionEmploye}\PYG{o}{(}\PYG{n}{identifiant}\PYG{o}{,} \PYG{n}{mdp}\PYG{o}{)}\PYG{o}{;}
\end{Verbatim}


\subsection{Lister tous les clients d'un employé}
\label{services:lister-tous-les-clients-d-un-employe}
\begin{Verbatim}[commandchars=\\\{\}]
\PYG{n}{Employe} \PYG{n}{employe} \PYG{o}{=} \PYG{n}{services}\PYG{o}{.}\PYG{n+na}{connexionEmploye}\PYG{o}{(}\PYG{n}{identifiant}\PYG{o}{,} \PYG{n}{mdp}\PYG{o}{)}\PYG{o}{;}

\PYG{k}{for}\PYG{o}{(}\PYG{n}{Client} \PYG{n}{element} \PYG{o}{:} \PYG{n}{employe}\PYG{o}{.}\PYG{n+na}{getClients}\PYG{o}{(}\PYG{o}{)} \PYG{o}{)}
\PYG{o}{\PYGZob{}}
    \PYG{n}{System}\PYG{o}{.}\PYG{n+na}{out}\PYG{o}{.}\PYG{n+na}{println}\PYG{o}{(}\PYG{n}{element}\PYG{o}{)}\PYG{o}{;}
\PYG{o}{\PYGZcb{}}
\PYG{c+c1}{// On peut vérifier qu\PYGZsq{}il a bien des clients}
\PYG{k}{if}\PYG{o}{(}\PYG{n}{choices}\PYG{o}{.}\PYG{n+na}{isEmpty}\PYG{o}{(}\PYG{o}{)}\PYG{o}{)}
\PYG{o}{\PYGZob{}}
    \PYG{n}{System}\PYG{o}{.}\PYG{n+na}{out}\PYG{o}{.}\PYG{n+na}{println}\PYG{o}{(}\PYG{l+s}{\PYGZdq{}Vous n\PYGZsq{}avez pas de clients !\PYGZdq{}}\PYG{o}{)}\PYG{o}{;}
\PYG{o}{\PYGZcb{}}
\end{Verbatim}

\begin{notice}{tip}{Astuce:}
La service \emph{services.listerHistoriqueClient(client)} permet de récupérer un historique préformaté. Pour plus de souplesse, vous pouvez directement utiliser la liste des horoscopes du client, ou encore utiliser les attributs de la classe \emph{horoscope}.
\end{notice}


\subsection{Afficher l'historique des prédictions}
\label{services:afficher-l-historique-des-predictions}
\begin{Verbatim}[commandchars=\\\{\}]
    \PYG{c+c1}{// Première Méthode}
    \PYG{n}{List}\PYG{o}{\PYGZlt{}}\PYG{n}{Horoscope}\PYG{o}{\PYGZgt{}} \PYG{n}{historique} \PYG{o}{=} \PYG{n}{services}\PYG{o}{.}\PYG{n+na}{listerHistoriqueClient}\PYG{o}{(}\PYG{n}{client}\PYG{o}{)}\PYG{o}{;}

    \PYG{c+c1}{//2ème méthode}
\PYG{k+kt}{boolean} \PYG{n}{vide} \PYG{o}{=} \PYG{k+kc}{true}\PYG{o}{;}
\PYG{k}{for}\PYG{o}{(}\PYG{n}{Horoscope} \PYG{n}{h} \PYG{o}{:} \PYG{n}{client}\PYG{o}{.}\PYG{n+na}{horoscopes}\PYG{o}{)}
\PYG{o}{\PYGZob{}}
    \PYG{k}{if}\PYG{o}{(}\PYG{n}{h} \PYG{o}{!}\PYG{o}{=} \PYG{k+kc}{null}\PYG{o}{)}
    \PYG{o}{\PYGZob{}}
        \PYG{n}{vide} \PYG{o}{=} \PYG{k+kc}{false}\PYG{o}{;}
        \PYG{n}{System}\PYG{o}{.}\PYG{n+na}{out}\PYG{o}{.}\PYG{n+na}{println}\PYG{o}{(}\PYG{l+s}{\PYGZdq{}\PYGZus{}\PYGZus{}\PYGZus{}\PYGZus{}\PYGZus{}\PYGZus{}\PYGZus{}\PYGZus{}\PYGZus{}\PYGZus{}\PYGZus{}\PYGZus{}\PYGZus{}\PYGZdq{}}\PYG{o}{)}\PYG{o}{;}
        \PYG{n}{System}\PYG{o}{.}\PYG{n+na}{out}\PYG{o}{.}\PYG{n+na}{println}\PYG{o}{(}\PYG{l+s}{\PYGZdq{}\PYGZbs{}n\PYGZbs{}nLe \PYGZdq{}}\PYG{o}{+}
        \PYG{n}{h}\PYG{o}{.}\PYG{n+na}{getDateHoroscope}\PYG{o}{(}\PYG{o}{)}\PYG{o}{.}\PYG{n+na}{get}\PYG{o}{(}\PYG{n}{Calendar}\PYG{o}{.}\PYG{n+na}{DAY\PYGZus{}OF\PYGZus{}MONTH}\PYG{o}{)}\PYG{o}{+} \PYG{l+s}{\PYGZdq{} \PYGZdq{}}
                    \PYG{o}{+} \PYG{l+s}{\PYGZdq{}\PYGZdq{}}\PYG{o}{+}\PYG{n}{h}\PYG{o}{.}\PYG{n+na}{getDateHoroscope}\PYG{o}{(}\PYG{o}{)}\PYG{o}{.}\PYG{n+na}{getDisplayName}\PYG{o}{(}
                    \PYG{n}{Calendar}\PYG{o}{.}\PYG{n+na}{MONTH}\PYG{o}{,}\PYG{n}{Calendar}\PYG{o}{.}\PYG{n+na}{LONG}\PYG{o}{,}\PYG{n}{Locale}\PYG{o}{.}\PYG{n+na}{FRANCE}\PYG{o}{)}\PYG{o}{+}\PYG{l+s}{\PYGZdq{}\PYGZdq{}}
                    \PYG{o}{+} \PYG{l+s}{\PYGZdq{} \PYGZdq{}}\PYG{o}{+}\PYG{n}{h}\PYG{o}{.}\PYG{n+na}{getDateHoroscope}\PYG{o}{(}\PYG{o}{)}\PYG{o}{.}\PYG{n+na}{get}\PYG{o}{(}\PYG{n}{Calendar}\PYG{o}{.}\PYG{n+na}{YEAR}\PYG{o}{)}\PYG{o}{)}\PYG{o}{;}
        \PYG{n}{System}\PYG{o}{.}\PYG{n+na}{out}\PYG{o}{.}\PYG{n+na}{println}\PYG{o}{(}\PYG{n}{h}\PYG{o}{)}\PYG{o}{;}
        \PYG{n}{System}\PYG{o}{.}\PYG{n+na}{out}\PYG{o}{.}\PYG{n+na}{println}\PYG{o}{(}\PYG{l+s}{\PYGZdq{}\PYGZbs{}n\PYGZbs{}n By \PYGZdq{}}\PYG{o}{+}\PYG{n}{h}\PYG{o}{.}\PYG{n+na}{getNomMedium}\PYG{o}{(}\PYG{o}{)}\PYG{o}{)}\PYG{o}{;}
        \PYG{n}{System}\PYG{o}{.}\PYG{n+na}{out}\PYG{o}{.}\PYG{n+na}{println}\PYG{o}{(}\PYG{l+s}{\PYGZdq{}\PYGZus{}\PYGZus{}\PYGZus{}\PYGZus{}\PYGZus{}\PYGZus{}\PYGZus{}\PYGZus{}\PYGZus{}\PYGZus{}\PYGZus{}\PYGZus{}\PYGZus{}\PYGZdq{}}\PYG{o}{)}\PYG{o}{;}
    \PYG{o}{\PYGZcb{}}
\PYG{o}{\PYGZcb{}}
\PYG{k}{if}\PYG{o}{(}\PYG{n}{vide}\PYG{o}{)}
\PYG{o}{\PYGZob{}}
    \PYG{n}{System}\PYG{o}{.}\PYG{n+na}{out}\PYG{o}{.}\PYG{n+na}{println}\PYG{o}{(}\PYG{l+s}{\PYGZdq{}Historique vide\PYGZdq{}}\PYG{o}{)}\PYG{o}{;}
\PYG{o}{\PYGZcb{}}
\end{Verbatim}


\subsection{Ajouter une prédiction à un horoscope}
\label{services:ajouter-une-prediction-a-un-horoscope}
\begin{Verbatim}[commandchars=\\\{\}]
    \PYG{c+c1}{// on liste une catégorie de prédictions}
\PYG{n}{List}\PYG{o}{\PYGZlt{}}\PYG{n}{PredictionAmour}\PYG{o}{\PYGZgt{}} \PYG{n}{pAmour} \PYG{o}{=} \PYG{n}{services}\PYG{o}{.}\PYG{n+na}{listerPredictionAmours}\PYG{o}{(}\PYG{o}{)}\PYG{o}{;}
\PYG{n}{Horoscope} \PYG{n}{horoscope} \PYG{o}{=} \PYG{k}{new} \PYG{n}{horoscope}\PYG{o}{(}\PYG{o}{)}\PYG{o}{;}
\PYG{n}{horoscope}\PYG{o}{.}\PYG{n+na}{setAmour}\PYG{o}{(}\PYG{n}{pAmour}\PYG{o}{.}\PYG{n+na}{get}\PYG{o}{(}\PYG{l+m+mi}{0}\PYG{o}{)}\PYG{o}{)}\PYG{o}{;} \PYG{c+c1}{// si on veut la première de la liste}
\end{Verbatim}


\subsection{Mettre la date de l'horoscope et l'ajouter à un client}
\label{services:mettre-la-date-de-l-horoscope-et-l-ajouter-a-un-client}
\begin{Verbatim}[commandchars=\\\{\}]
\PYG{n}{Calendar} \PYG{n}{date} \PYG{o}{=} \PYG{n}{Calendar}\PYG{o}{.}\PYG{n+na}{getInstance}\PYG{o}{(}\PYG{o}{)}\PYG{o}{;}
\PYG{n}{horoscope}\PYG{o}{.}\PYG{n+na}{setDateHoroscope}\PYG{o}{(}\PYG{n}{date}\PYG{o}{)}\PYG{o}{;}
\PYG{n}{services}\PYG{o}{.}\PYG{n+na}{creerHoroscope}\PYG{o}{(}\PYG{n}{horoscope}\PYG{o}{,} \PYG{n}{client}\PYG{o}{)}\PYG{o}{;}
\end{Verbatim}


\chapter{Présentation des maquettes}
\label{maquette:presentation-des-maquettes}\label{maquette::doc}


\renewcommand{\indexname}{Index}
\printindex
\end{document}
